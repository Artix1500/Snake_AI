\documentclass[]{article}

%opening
\title{}
\author{Karolina Drabent \\ Patryk Fijałkowski \\ Artur Haczek \\ Krzysztof Kamiński \\ Pamela Krzypkowska}



\begin{document}

\maketitle

\begin{abstract}
	
\par This short paper will cover the topic of Q-Learning tested on an Agent base input game, which in our scenario is a simple Snake game. We will first explain what Q-Learning is, how and why it works and what is it used for. Later we will present our solution with the usage of Q-Learning methods which goal is to teach an Agent to play our Snake game, maximise its score and amount of moves per game. Finally we would show how our method is working in practise, what changes could be made further on as well as different approaches that we had encountering numerous problems along with their actual outcomes.

\end{abstract}

\section{Background Knowledge }

\end{document}
